\chapter{Risikomanagement}
\label{sec:risikomgmt}

Die Risiken, die von einem Medizinprodukt ausgehen, sind jederzeit bekannt und beherrscht. Die Risiken sind im Vergleich zum Nutzen des Medizinproduktes akzeptabel. Das Risikomanagement für ein Medizinprodukt beginnt mit dem Start der Entwicklung des neuen Produktes und endet erst dann, wenn dieses Produkt auf dem Markt nicht mehr verwendet wird. Somit können zwei Hauptphasen des Risikomanagements unterschieden werden:
\begin{description}
	\item[Entwicklungsbegleitendes Risikomanagement] \hfil \\
	Das Risikomgmt während der Entwicklung hat das Ziel sämtliche Risiken, welche von einem Produkt ausgehen zu erfassen und zu beherrschen. Das neue Produkt wird bei der Einführung nicht frei von Risiken sein, aber die verbleibenden Risiken sind bekannt und in einem akzeptablen Verhältnis zum Nutzen des Produktes.
	
	\item[Risikomanagement nach der Markteinführung] \hfil \\
	Sobald das neue Produkt für den Markt freigegeben wird, muss das Produkt bezüglich Risiken weiterhin systematisch überwacht werden. Solange das Produkt auf dem Markt noch verwendet wird, muss diese Überwachungstätigkeit durchgeführt und evtl. Änderungen am Produkt vorgenommen werden. Der Produktlebenszyklus dauert von der Produktentwicklung (z.B. 2015) bis das letzte Produkt ausser Betrieb genommen wurde (2040).
\end{description}

\section{Gebrauchstauglichkeit von Medizinprodukten}

Ein wichtiger Faktor um Risiken mit Medizinprodukten zu vermeiden ist die Gebrauchstauglichkeit. Ein Medizinprodukt sollte folgende Eigenschaften aufweisen, um als gebrauchstauglich zu gelten:
\begin{itemize}
	\item Effektivität (Kann das Ziel mit dem Produkt vollständig erreicht werden?)
	\item Effizienz	(Mit welchem Aufwand wird dieses Ziel erreicht?)
	\item Zufriedenheit	(Welche Reaktion löst das Produkt beim Anwender aus – stört oder unterstützt es?)
\end{itemize}
Zahlreiche Umfragen zeigen aber dass diese Eigenschaften meist nicht vorhanden sind, was ein erhebliches Risiko für den Patienten darstellt (z.B. 44'000 – 98'000 Todesfälle durch Gebrauchstauglichkeitsprobleme in USA). Die Entwicklung von gebrauchstauglichen Produkten ist eine interdisziplinäre Aufgabe, bei der die Mensch-Maschinen-Schnittstelle möglichst parallel zur Technologie entwickelt werden sollte. Man sollte den Anwender nicht nur befragen, sondern auch bei seiner Arbeit beobachten und ihn nicht mit Funktionen und Komplexität eines Gerätes überfluten. Folgender Prozess kann ausgeführt werden um ein benutzerfreundliches Produkt zu erhalten:
\begin{enumerate}
	\item Feststellen der Notwendigkeit einer benutzerorientierten Gestaltung
	\item Verstehen und Festlegen des Nutzungskontexts (z.B. Fokusgruppen, Feldbeobachtungen, Cognitive Walkthrough)
	\item Festlegen von Benutzeranforderungen und organisatorischen Anforderungen (Primär-/Sekundärfunktion)
	\item Entwerfen von Gestaltungslösungen (Prototyp)
	\item Verifizierung und Validierung	der Gebrauchstauglichkeit durch Nutzertests. Wenn ok fertig sonst alles nochmal wiederholen.
\end{enumerate}

\section{Risikomanagement-Prozess}

Die ISO 13485 verweist beim Risikomgmt auf die Norm ISO 14971, welche einen Prozess zum Risikomgmt vorschlägt. Dieser wird nachfolgend beschrieben.

\subsection{Risikoanalyse}

Der Ausgangspunkt der Risikoanalyse ist die Zweckbestimmung, welcher zu Beginn der Produktentwicklung festgelegt wird. Zusätzlich zur Zweckbestimmung muss aber auch bestimmt werden, wie das Produkt möglicherweise Missbraucht werden könnte. Danach sollen Gefährdungen identifiziert werden die vom Produkt ausgehen können. Auf diese Gefährdungen müssen dann z.B. in der Bedienungsanleitung hingewiesen werden. Die Beschreibung einer Gefährdung beinhaltet den Anwender/Patient nicht. Danach wird die Gefährdungssituation, also die Abfolge von Ereignissen, die von der Gefährdung über die Gefährdungssituation zum Schaden führen, detailliert beschrieben. In diese Beschreibung wird der Anwender/Patient miteinbezogen.

Anhand einer Risikomatrix wird jede Gefährdungssituation eingeschätzt. Die Risikomatrix legt das Unternehmen selbst fest. Allerdings besteht sie immer aus einer Achse mit der Auftretenswahrscheinlichkeit und mit einer Achse über den Schweregrad der Auswirkung. Ist das Risiko akzeptabel müssen keine Massnahmen ergriffen werden. Jede Gefährdungssituation wird in der Risikomatrix eingetragen. Ist das Risiko inakzeptabel müssen Massnahmen zur Risikominderung getroffen werden. Ein Risiko kann auch im ALARP-Bereich (\textit{as low a reasonable possible}) liegen. Hier sollen Massnahmen nur soweit sinnvoll umgesetzt werden. Viele Unternehmen besitzen aber keinen ALARP-Bereich.

\subsection{Risikobewertung}

Wenn eine Risikomatrix wie oben beschrieben verwendet wird, dann ist die Risikobewertung sehr einfach, da die Risiko- und Akzeptanzkriterien bereits in der Risikomatrix dargestellt sind: Für alle Gefährdungssituationen, die nicht als akzeptabel eingestuft wurden, muss jeweils eine Risi-kominderung erarbeitet werden.

\subsection{Risikobeherrschung}

Für alle Gefährdungssituationen, die nicht im grünen Bereich der Risikomatrix liegen, müssen Massnahmen zur Risikoreduktion erarbeitet werden. Dabei ist zwingend folgende Reihenfolge einzuhalten:
\begin{enumerate}
	\item Integrierte Sicherheit durch Design (z.B. scharfe Kanten brechen)
	\item Schutzmassnahmen (z.B. Skalpell mit Schutzhülle abdecken)
	\item Informationen zur Sicherheit (z.B. Piktogramm auf Verpackung)
\end{enumerate}
Sind Massnahmen erarbeitet worden, muss nachgewiesen werden dass diese tatsächlich umgesetzt und Risiko auch verringern.

\subsection{Bewertung der Akzeptanz des Gesamt-Restrisikos}

Nach der Umsetzung der Massnahmen, wird nochmals eine Risikobewertung durchgeführt. Es sollte jetzt kein Risiko mehr im roten Bereich sein. Falls doch kann der Hersteller abwägen ob der medizinische Nutzen höher ist als das Risiko und das Risiko trotzdem akzeptieren. Risiken im gelben Bereich (ALARP) müssen nochmals untersucht werden und es muss entschieden werden, ob das Restrisiko weiter zu senken ist.

\subsection{Risikomanagementbericht}

Der Hersteller des Medizinproduktes muss die Ergebnisse des Risikomanagementprozesses vor der Produktfreigabe überprüfen. Diese Überprüfung muss dokumentiert werden.

\subsection{Informationen aus der Herstellung und der Herstellung nachgelagerten Phasen}

Nachdem das Medizinprodukt auf dem Markt ist müssen Hinweise von Anwendern, der Produktion oder auch überarbeitete Normen überwacht werden und bei Bedarf das Produkt angepasst werden.