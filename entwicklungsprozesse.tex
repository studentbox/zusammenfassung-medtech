\chapter{Entwicklungsprozesse für Medizinprodukte}
\label{sec:produktentwicklung}

Die ISO 13485 schreibt vor dass ein Hersteller eines Medizinproduktes einen definierten Entwicklungsprozess haben muss. Nachfolgend wird der Entwicklungsprozess der Firma B. Braun beschrieben.

\section{Ermittlung der Anforderungen (Projektstart)}

Aus einer Produktidee entsteht ein Projekt. Nach dem Projektstart muss die Projektorganisation, die Produktbeschreibung und ein Lastenheft vorliegen. Um diese Artefakte zu erstellen müssen die expliziten und impliziten Anforderungen der Kunden und die des Gesetzgebers ermittelt werden. Dies geschieht meist in der Form einer Zweckbestimmung.

\section{Design Input (Konzeptionierung)}

Aufgrund des Lastenhefts werden technische Lösungskonzepte erstellt und in einem Pflichtenheft (Produktkonzept, Entwicklungsspezifikation) festgehalten.

\section{Design Output (Design)}

Die Konzepte welche im Pflichtenheft beschrieben wurden, werden nun in konkrete Lösungen (z.B. Konstruktionszeichnungen, Materialspezifikationen, Produktionsspezifikationen) umgesetzt. Dabei kann man nicht irgendeine Lösung entwerfen, sondern muss immer belegen können, dass man sich an die Vorgaben gehalten hat.

\section{Verifizierung der Entwicklungsergebnisse (Produktqualifikation)}

In der Produktequalifikation wird nun geprüft ob das Produkt, dass z.B. als Prototyp vorliegt, den technischen Vorgaben aus dem Design Output entspricht. Dabei müssen die Ergebnisse der Verifizierung in technischen Testberichten festgehalten werden.

\section{Validierung des Produktes (Anwendertest)}

Nachdem die Verifizierung erfolgreich war muss mit Anwendertests nachgewiesen werden, dass das Produkt den Kundenanforderungen gerecht wird. Für die Validierung und Verifizierung der Produkte wird das V-Modell aus der Informatik verwendet.

\section{Design Transfer}

Nach der Validierungphase wird das Produkt von der Produktentwicklung an die Prozessentwicklung übergeben. Die Prozessentwicklung kümmert sich um die Anlagenbeschaffung, die Produktionsvalidierung und die Markteinführung. 

Während des Entwicklungsprozesses müssen gelegentlich Reviews der Prozessschritte durchgeführt und dokumentiert werden. Möchten Änderungen am Produkt vorgenommen werden, können diese nicht einfach willkürlich durchgeführt werden. Es muss ein definiert Änderungsprozess implementiert werden und der ganze Verifizierungs-/Validierungszyklus nochmals durchlaufen werden.