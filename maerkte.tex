\chapter{Märkte für Medizinprodukte}

Die Medizintechnik beschreibt ein Fachgebiet des Life Science Engineerings, welches sich am Schnittpunkt von Medizin und Ingenieurswissenschaften befindet. Es geht dabei um das Übertragen von technischem Wissen auf medizinische Untersuchungsmethoden, Therapieverfahren, Krankenpflege, Rehabilitation, und andere Gebiete. Ziel ist dabei stets die Optimierung der Behandlungsergebnisse des Patienten, sowie die Erleichterung der Arbeit für das medizinische Fachpersonal.

\section{Weltweite Trends im Gesundheitswesen}

Die drei Segemente mit dem grössten Wachstum in der Medizintechnik ist die Blutdiagnose (In Vitro Diagnostics), Herz \& Kreislauf (Cardiology) und Augen (Ophthalmics). Das weltweit grösste Wachstum (77\%) hat die Medizintechnik in der BRIC-Zone (Brazil, Russia, India, China). Märkte wie die USA und Europa bieten kaum mehr Wachstum (4\%) und das zu einem hohen Risiko. Die Bedürfnisse der Schwellenländer lassen sich in drei Segmente unterteilen:

\begin{description}
	\item[Globale Premium Produkte] \hfil \\ 
		\begin{itemize}
			\item Wohlhabende Konsumenten (Grösste Kaufkraft)
			\item Zugang zu hochwertiger Gesundheitsfürsorge
			\item Hohe Kosten/High-End-Lösungen
			\item Westliche Technologie/Angebote
		\end{itemize}
	\item[Lokale Produkte] \hfil \\ 
		\begin{itemize}
			\item Wachsendes Mittelklasse-Segment
			\item Eingeschränkter Zugang zu Gesundheitsfürsorge
			\item Rentable/Erschwingliche Lösungen (Produktion in den Schwellenländern)
			\item Angepasste Technologie/Angebote
		\end{itemize}
	\item[Disruptove billige Produkte] \hfil \\ 
		\begin{itemize}
			\item Ländliche Konsumenten
			\item Keinen oder sehr limitierten Zugang zu Gesundheitsfürsorge
			\item Günstigste Lösungen
			\item Disruptive Angebote (Innovative Ansätze welche den Markt verändern)
		\end{itemize}
\end{description}
Die weltweiten Trends im Gesundheitswesen sind nachfolgend stichwortartig aufgelistet:

\begin{itemize}
	\item Weniger Behandlungen (Stagnierende Wirtschaft)
	\item Knappere Staatshaushalte (Kleinere Gesundheitsbudgets)
	\item Kostendruck auf Gesundheitsanbieter (Kaufen vermehrt gebündelte Angebote)
	\item  Weniger invasive Behandlungen (Medikamente statt chirurgische Behandlung)
	\item Strengere Regulierungen (Entwicklungskosten eskalieren, insbesondere für kleine Unternehmen)
	\item Neue Technologien (Technologie-Wettkampf bleibt intensiv)
	\item Personalisierte Medizin (Behandlung aufgrund von \textit{genetischen} Analysen)
\end{itemize}

\section{Technologietrends}

Zudem gibt es fünf Innovationsrichtungen, welche nachfolgend aufgelistet sind:

\begin{description}
	\item[Miniaturisierung:] Verkleinerung technischer Komponenten und Systeme
	\item[Biologisierung:] Verschmelzung biologischer und technischer Komponenten
	\item[Computerisierung:] Integration von Informations- und Kommunikationstechnik in Medizintechnik
	\item[Personalisierung:] Behandlung abgestimmt auf Patient
	\item[Vernetzung:] Integration von Medizinprodukten in bestehende Datennetzwerke
\end{description}

\section{Medizintechnik in der Schweiz}

Die Schweiz hat die grösste Dichte an Medizintechniker auf die Einwohner gerechnet. Die Medizintechnik-Branche besteht aus ca. 1'500 Unternehmen (Hersteller, Zulieferer, Dienstleister und Händler), von welchen der grösste Anteil KMUs sind (< 250 Angestellte). Mit 2.3\% Anteil am BIP die Schweiz führend unter den Medtech-Ländern. Das Exportvolumen beträgt 5\% der gesamten Exporte der Schweiz. 