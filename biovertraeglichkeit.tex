\chapter{Biokompatibilität}

Wenn es zu Problemen oder Rückrufaktionen von Implantaten kommt, kann das zu einem Imageverlust oder sogar zu Freiheitsstrafen führen (z.B. vorsätzlich Industriesilikon für Brustimplantat genutzt). Bei Implantaten sind immer Biomaterialien im Spiel. Als \textit{Biomaterial} oder zum Teil als \textit{Implantatmaterial} werden allgemein synthetische oder nichtlebende natürliche Materialien oder Werkstoffe bezeichnet, die in der Medizin für therapeutische oder diagnostische Zwecke eingesetzt werden und dabei in unmittelbaren Kontakt mit biologischem Gewebe des Körpers kommen. Diese Materialien treten dabei in chemische, physikalische und biologische Wechselwirkungen mit den entsprechenden biologischen Systemen. Das Ziel wäre ein Material zu finden das überhaupt nicht mit dem Körper reagiert. Es wird aber immer eine Reaktion geben, deshalb ist es wichtig eine angemessene Reaktion zu finden. Wenn ein Werkstoff also für eine spezifische Anwendung eine angemessene Reaktion zeigt, um eine bestimmte Funktion auszuüben, ist er Biokompatibel. Tabelle \ref{tab:wechselwirkung-biomaterial-gewebe} zeigt wie ein Implantat mit dem Gewebe reagieren kann.

\begin{table}
	\centering
	\small
	\renewcommand{\arraystretch}{1.5}
	\begin{tabular}{|l|p{10cm}|}
		\hline Implantat-Eigenschaften & Gewebereaktion \\ 
		\hline toxisch & Gewebenekrose (Gewebe stirbt ab) \\ 
		\hline inert & Gewebe bildet eine Bindegewebskapsel um das Implantat (Implantat kann nicht mehr mit dem Gewebe reagieren) \\ 
		\hline bioaktiv & Gewebe bildet eine Bindung mit dem Implantat aus (Knochen verbindet sich mit Titan von künstlichem Hüftgelenk) \\ 
		\hline degradabel & Gewebe ersetzt Implantat (z.B. Fäden die sich auflösen) \\ 
		\hline 
	\end{tabular} 
	\caption{Wechselwirkung Biomaterial - Gewebe}
	\label{tab:wechselwirkung-biomaterial-gewebe}
\end{table}

Als biokompatible Werkstoffe können Metalle, Polymere, Faserverbundwerkstoffe und keramische Werkstoffe verwendet werden. Bei den Metallen werden meist rostfreie Stähle (Implantate, chirurgische Instrumente), Kobaltlegierungen (Gelenkersatz, Fixation von Knochenbrüchen) und Titanlegierungen (Implantate, chirurgische Instrumente) verwendet. Von Medizinprodukten können kurzzeitig folgende biologischen Gefährdungen ausgehen:
\begin{description}
	\item[Akute Toxizität] Giftige Wirkung auf Zellen, Leber usw. aufgrund einmaliger Einwirkung
	\item[Irritation] Haut-, Augenreizung oder Entzündungsreaktion
	\item[Hämolyse] Auflösung der roten Blutkörperchen
	\item[Thrombogenität] Bildung eines Blutgerinnsels. Oberfläche des Implantates wirkt auf das Blut wie eine Verletzung.
\end{description}
Zudem gibt es diverse Gefährdungen die erst nach einer gewissen Zeit auftreten:
\begin{description}
	\item[Chronische Toxizität] Giftige Wirkung auf Zellen, Leber usw. aufgrund kontinuierlicher Einwirkung
	\item[Allergie] Überreaktion des Immunsystems z.B. Pollenalergie
	\item[Sensibilisierung] Aufbau einer fehlgeleiteten spezifischen Immunantwort bei Erstkontakt. Zweitkontakt mit Werkstoff kann Allergie auslösen z.B. Nickelallergie.
	\item[Genotoxizität] Veränderungen im genetischen Material
	\item[Karzinogenität] Krebserregend z.B. Weichmacher im Kunststoff
	\item[Reproduktionstoxizität] Beeinträchtigung der Fortpflanzungsfähigkeit oder Schädigung des Embryos
\end{description}